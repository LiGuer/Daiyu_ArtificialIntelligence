\documentclass{article} 
\usepackage{latexsym}
\usepackage{amssymb}
\usepackage[UTF8]{ctex}
\usepackage{geometry}\geometry{left=2cm, right=2cm, top=2cm, bottom=2cm}
\usepackage{framed} 
\setlength{\parindent}{0pt}
\title{Physics}\author{LiGu 梨菇}\date{}
\begin{document}
\maketitle

\section{Mechanics力学}
\subsection{力学系统描述}
(对于s个自由度的系统,)\\
\textbf{广义坐标}: 完全刻画其位置的任意s个变量 $q_{1},q_{2},\cdots,q_{s}$\\
\textbf{广义速度}: 广义坐标对时间的一阶导$\dot q_{1},\dot q_{2},\cdots,\dot q_{s}$\\
· 经验表明, 同时给定广义坐标、广义速度, 就可确定系统状态, 原则上可预测以后动作.\\
(加速度$\ddot q$, 可由$\dot q,q$唯一确定.)\\
\textbf{运动方程}: 加速度与坐标、速度的关系式. (二阶微分方程,原则上积分得q(t)可确定系统运动轨迹.)\\
\textbf{Lagrange函数}: 每一个力学系统可以用一个确定函数L()表征.
\[L(q_{1},q_{2},\cdots,q_{s},\dot q_{1},\dot q_{2},\cdots,\dot q_{s},t)\]
\textbf{最小作用量原理}: 系统在两个时刻的, 位置之间的运动, 使得Lagrange函数积分(作用量S)取最小值.
\[S = \int_{t1}^{t2} L(q,\dot q,t)dt\]
\[\delta S = \delta \int_{t1}^{t2} L(q,\dot q,t)dt
 = \int_{t1}^{t2} (\frac{\partial L}{\partial q}\delta q + \frac{\partial L}{\partial \dot q}\delta \dot q) dt
 = \frac{\partial L}{\partial \dot q} \delta q \arrowvert_{t1}^{t2}
 + \int_{t1}^{t2} (\frac{\partial L}{\partial q} - \frac{d}{dt} \frac{\partial L}{\partial \dot q}) \delta q dt
 = 0\]
\textbf{Lagrange方程}: 运动微分方程. 
\[\Rightarrow \frac{d}{dt}\frac{\partial L}{\partial \dot q_i} - \frac{\partial L}{\partial q_i} = 0\quad(i=1,\dots,s)\]


\subsection{参考系}
· 研究力学现象必须选择参考系.\\
\textbf{惯性参考系}:空间相对它均匀且各向同性,时间相对它均匀.(特别的,某时刻静止的自由物体将永远保持静止.)\\
$\Rightarrow$ Lagrange函数不显含$\vec r,t$,不依赖$\vec v$的矢量方向,即
\[L = L(v^2)\]
$\Rightarrow$ Lagrange方程有$\frac{d}{dt}\frac{\partial L}{\partial \vec v} = -\frac{\partial L}{\partial \vec r} = 0 \Rightarrow \frac{\partial L}{\partial \vec v}=const$\\
\textbf{惯性定律}: 在惯性参考系中,质点任何自由运动的速度的大小和方向都不改变.
\[\Rightarrow \vec v = const\]


\subsection{参考系间相对性}
\textbf{Galilean相对性原理}: 存在无穷个相互作匀速直线运动的惯性参考系,这些惯性系之间时空性质相同,所有力学规律等价.\\
(今后不特别声明,默认惯性参考系.)\\
\textbf{Galilean变换}: 两个不同参考系K、K'之间的的坐标变换(K'相对K以速度$\vec V$运动).\\
\textbf{绝对时间假设}: 我们认为两个参考系的时间相同.
\begin{displaymath}
    \left\{ \begin{array}{ll}
    \vec r = \vec r' + \vec V t\\
    t = t'
    \end{array} \right.
\end{displaymath}
\textbf{Galilean相对性原理}: 力学系统在Galilean变换下具有不变性.


\subsection{质点、质点系}
\textbf{质点的Lagrange方程}:
\textbf{质量 m}:
\[L = \frac{m}{2} v^2\]
\textbf{质点系的Lagrange方程}:\\
\textbf{动能 T}: \quad \textbf{势能 U}:描述质点之间相互作用,而增加的关于坐标的函数(由相互作用性质决定)(限于经典力学).
\[L = \sum \frac{m_i v_i^2}{2} - U(\vec r_1,\vec r_2,\cdots ) = T(v_1^2,v_2^2,\cdots) - U(\vec r_1,\vec r_2,\cdots )\]



\subsection{守恒定律}
\subsubsection{能量守恒 : 时间均匀性}
时间均匀性: 封闭系统Lagrange函数不显含时间.\quad $\Rightarrow L(\vec q,\dot \vec q)$\\
\[\frac{dL}{dt}
 = \sum \frac{\partial L}{\partial q_i} \dot q_i + \sum \frac{\partial L}{\partial \dot q_i} \ddot q_i
 = \sum \frac{d}{dt}(\frac{\partial L}{\partial \dot q_i}\dot q_i)
\Rightarrow \frac{d}{dt}(\sum \dot q_i \frac{\partial L}{\partial \dot q_i} - L) = 0\]
\textbf{能量E}: 
\[\Rightarrow E = \sum \dot q_i \frac{\partial L}{\partial \dot q_i} - L  = T(q,\dot q) + U(q) = const.\]


\subsubsection{动量守恒 : 空间均匀性 , \quad 力}
空间均匀性: 空间平移不变性.\\
\[\delta L = \sum \frac{\partial L}{\partial \vec r_i}\cdot \delta \vec r_i
 = \vec \epsilon \cdot \sum \frac{\partial L}{\partial \vec r_i} = 0\quad(\forall \vec \epsilon)
\Rightarrow \sum \frac{\partial L}{\partial \vec r_i}
 = \sum \frac{d}{dt} \frac{\partial L}{\partial \vec v_i}
 = \frac{d}{dt} \sum \frac{\partial L}{\partial \vec v_i} = 0\]
\textbf{动量$\vec P$: }
\[\Rightarrow \vec P = \sum \frac{\partial L}{\partial \vec v_i} = \sum m_i \vec v_i = const.\]
\textbf{力$\vec F$}:
\[\Rightarrow \sum \frac{\partial L}{\partial \vec r_i} = \sum \vec F_i = 0 \quad , \quad \vec F = \frac{\partial L}{\partial \vec r}\]
\textbf{广义动量}:
\[p_i = \frac{\partial L}{\partial \dot q_i}\]
\textbf{广义力}:
\[F_i = \frac{\partial L}{\partial \dot q_i} \quad , \quad \dot P_i = F_i\]

\subsubsection{角动量守恒 : 空间各向同性}
空间各向同性: 空间旋转不变性.
\begin{displaymath}
\begin{array}{ll}
\delta L = \sum (\frac{\partial L}{\partial \vec r_i} \cdot \delta \vec r_i + \frac{\partial L}{\partial \vec v_i} \cdot \delta \vec v_i)
 = \sum [\dot{\vec P_i} \cdot (\delta \vec \varphi \times \vec r_i) + \vec P_i \cdot ( \delta \vec \varphi \times \vec v_i )]
 = \delta \vec \varphi \cdot \sum (\vec r_i \tims \dot{\vec P_i} + \vec v_i \tims \vec P_i) \\
 \quad\  = \delta \vec \varphi \cdot \frac{d}{dt} \sum \vec r_i \times \vec P_i
 = 0\quad(\forall \vec \varphi)
 \Rightarrow \frac{d}{dt}\sum \vec r_i \times \vec P_i = 0
\end{array}
\end{displaymath}
\textbf{角动量$\vec M$: }
\[\Rightarrow \vec M = \sum \vec r_i \times \vec P_i = const.\]


\subsubsection{$E,\vec P,\vec M$参考系间变换}
不同惯性参考系中(K'相对K以速度$\vec V$运动),\\
E 的变换:
\[E = \frac{1}{2} \sum m_iv_i^2 + U = \frac{1}{2} \sum m_i(\vec v_i + \vec V)^2 + U 
 = \frac{m_c V^2}{2} + \vec V \cdot \sum m_i \vec v'_i + \frac{1}{2} \sum m_i v'_i^2 + U
 = E' + \vec V \cdot \vec P' + \frac{m_c V^2}{2}\]
$\vec P$的变换:
\[\vec P = \sum m_i \vec v_i = \sum m_i \vec v'_i + \vec V \sum m_i = \vec P' + \vec V \sum m_i\]
$\vec M$的变换(K'相对K以速度$\vec V$运动且坐标原点相差$\vec R$):
\[\vec M = \sum m_i \vec r_i \times \vec v_i = \sum m_i (\vec r'_i + \vec R) \times (\vec v'_i + \vec V)
 = \sum m_i( \vec r'_i \times \vec v'_i +  \vec R \times \vec v'_i + \vec r'_i \times \vec V + \vec R \times \vec V)
 = \vec M' + \vec R \times \vec P'_c + m_c \vec r'_c \times \vec V + m_c \vec R \times \vec V\]


\subsection{力学相似性}
* Lagrange函数乘任意常数,不会改变运动方程.
\[\vec r_i \to \alpha \vec r_i, t \to \beta t 
\Rightarrow \vec v_i = \frac{d\vec r_i}{dt} \to \frac{\alpha}{\beta}\vec v_i,T \to \frac{\alpha^2}{\beta^2}T,U \to  \alpha^k U\]
若$\frac{\alpha^2}{\beta^2} = \alpha ^ k $,即$\beta = \alpha^{1-k/2}$,则Lagrange函数乘const.,运动方程不变.前后运动轨迹相似,只是尺寸不同.\\
且各力学量之比,满足:\quad(l:轨迹线度)
\[\frac{t'}{t} = (\frac{l'}{l})^{1-k/2},\frac{v'}{v} = (\frac{l'}{l})^{k/2},\frac{E'}{E} = (\frac{l'}{l})^k,\frac{M'}{M} = (\frac{l'}{l})^{1+k/2}\]\\
· 例1: 均匀力场,势能与坐标成线性,$\Rightarrow k=1,\Rightarrow \frac{t'}{t} = \sqrt{\frac{l'}{l}}$,重力场自由落体,下落时间平方与初始高度成正比.\\
· 例2: Kepler's第三定律,Newton引力、Coulomb力,势能与两点间距离成反比,$\Rightarrow k=-1,\Rightarrow \frac{t'}{t} = (\frac{l'}{l})^{3/2}$,轨道运动周期的平方与轨道尺寸的立方成正比.


\subsection{质心}
\textbf{质心系}:\exists 速度$\vec V$使得系统相对K'静止($\vec P' = 0$),且K'原点为系统质量中心,K'即质心系:
\[\vec V = \frac{\vec P}{\sum m_i} = \frac{\sum m_i \vec v_i}{\sum m_i}\]
\textbf{质心}: 
\begin{displaymath}
    \left\{ \begin{array}{ll}
    m_c = \sum m_i\\
    \vec v_c = \frac{\sum m_i \vec v_i}{\sum m_i}\\
    \vec r_c = \frac{\sum m_i \vec r_i}{\sum m_i}
    \end{array} \right.
\end{displaymath}
质心动量: $\vec P_c = m_c \vec v_c$\\
质心能量: \\
内能$E_{int}$: 整体静止的(质心系内)力学系统的能量,包括系统内相对运动动能 + 相互作用势能.\\
\[E = \frac{m_c V^2}{2} + \vec V \cdot \vec P' |_{\vec P'=0} + E_{int} = E_{int} + \frac{m_c v_c^2}{2}\]
质心角动量: 
\[\vec M = \vec M' + \vec R \times \vec P'_c + m_c \vec r'_c \times \vec V + m'_c \vec R \times \vec V|_{\vec P'_c =\vec r'_c = 0, \vec R =\vec r_c, \vec V =\vec v_c} = \vec M_{int} + \vec r_c \times \vec P_c \]
\textbf{质心组合关系}: 将质点系分成若干小系,各小系质心构成新的质点系之质心即为原质点系的质心.


\subsection{情景: 一维运动}
一维运动,定常外部条件下,Lagrange函数:
\[L = \frac{1}{2} = a(q) \dot q^2 - U(x)\quad (CartesianCoord)\Rightarrow L = \frac{m \dot x^2}{2} - U(x)\]
第一积分——能量守恒,有
\[E = \frac{\partial L}{\partial \dot q}\dot q - L = \frac{1}{2}m\cdot 2 \dot x \cdot \dot x - (\frac{m \dot x^2}{2} - U(x)) = \frac{m \dot x^2}{2} + U(x) \]
运动方程:
\[\dot x = \frac{dx}{dt} = \sqrt{\frac{2}{m}[E - U(x)]}\quad \Rightarrow \quad t = \sqrt{\frac{m}{2}} \int \frac{dx}{\sqrt{E - U(x)}} + const\]
$\because$动能恒正,故运动只能发生在$U(x) \leqslant E$的空间区域.




\subsection{情景: 二体问题}
\textbf{二体问题}: 两个相互作用的质点,组成的系统的运动.\\
· 相互作用的两个质点的势能仅依赖于它们之间的距离.\\
二体问题,Lagrange函数:
\[L = \frac{m \vec \dot r_1^2}{2} + \frac{m \vec \dot r_1^2}{2} - U(|\vec r_1 - \vec r_2|)\]
将问题分解为\textbf{质心运动}和\textbf{相对质心运动},以质心为原点:
\[m_1\vec r_1 + m_2 \vec r_2 =0,\quad \vec r_{12} = \vec r_1 - \vec r_2, \quad \Rightarrow \quad \vec r_1 = \frac{m_2}{m_1 + m_2}\vec r_{12}, \quad \vec r_2 = - \frac{m_1}{m_1 + m_2}\vec r_{12} \]
\[\Rightarrow L = \frac{m_{12} \vec \dot r_{12}^2}{2} - U(|\vec r_{12}|), \quad m_{12} = \frac{m_1 m_2}{m_1 + m_2}\]
于是,"二体问题"等效为一个质量$m_{12}$的质点,在外场$U(\vec r_{12})$下的运动, 而两个质点的运动$\vec r_1, \vec r_2$, 可由$\vec r_{12}$分别解出.


\subsection{情景: 有心力场}
\textbf{有心力场}: 质点势能只与质点到某一固定点的距离有关的外场.\\
\textbf{有心力}: 始终指向or背离与质点到某一固定点的方向,且大小只依赖r的力.
\[\vec F = -\frac{\partial U(r)}{\partial \vec r} = -\frac{d U(r)}{d r} \hat r\]
\textbf{[1]}: 有心力场角动量,Lagrange函数求解.\\
$\because$中心对称外场下(即势能仅依赖到空间某特定点(中心)距离),系统角动量在任意过中心的轴上投影都守恒.
\[\Rightarrow \vec M = \vec r \times \vec P = const.\]
$\Rightarrow$ 质点运动在垂直于$\vec M$的平面内.\quad $\Rightarrow$ 有心力场,Lagrange函数:
\[L = \frac{1}{2}m v^2 - U(r) = \frac{m}{2} (\dot r^2 + r^2 \dot \varphi ^2) - U(r)\]
\textbf{有心力场角动量守恒,值为$m r^2 \dot \varphi$}\\
$\varphi$的广义动量:
\[P_\varphi = \frac{\partial L}{\partial \dot \varphi} = m r^2 \dot \varphi \quad , \quad \frac{d P_\varphi}{d t} = \frac{d}{d t}\frac{\partial L}{\partial \dot \varphi} = \frac{\partial L}{\partial \varphi} \frac{d \frac{d \varphi}{d \varphi / d t}}{d t} = \frac{\partial L}{\partial \varphi} = 0 \quad , \quad |\vec M| = M_z = \sum \frac{\partial L}{\partial  \dot \varphi} = P_\varphi\]
\[\Rightarrow |\vec M| =| \vec r \times \vec P |= P_\varphi = m r^2 \dot \varphi = const.\]
\textbf{Kepler's第二定律}:质点矢径在相同时间内扫过的面积相等.
\[\Rightarrow M = m r^2 \dot \varphi = 2 m \dot S_{sector} = const.\quad \Rightarrow \dot S_{sector} = \frac{1}{2} r \cdot r d\varphi = const.\]
\textbf{[2]}: 有心力场运动方程求解.\\
能量守恒,有
\[E = \frac{\partial L}{\partial \dot q}\dot q - L = \frac{m}{2} (\dot r^2 + r^2 \dot \varphi ^2) + U(r) = \frac{m \dot r^2}{2} + \frac{M^2}{2mr^2} + U(r)\]
运动方程,有
\[\Rightarrow t = \int \frac{d r}{\sqrt{\frac{2}{m}[E-U(r)] - \frac{M^2}{m^2 r^2}}} + const. \quad \varphi = \int \frac{M}{m r^2} d t  + const.= \int \frac{M dr}{\sqrt{2mr^2 [E-U(r)] - \frac{M^2}{m^2 r^2}}} + const.\]
\textbf{[3]}: 有心力场径向运动,和一维运动的联系.\\
\textbf{等效势能}
\[U_{eff} = U(r) + \frac{M^2}{2mr^2}\]
\textbf{离心势能}
\[U_{centrifuge} = \frac{M^2}{2mr^2}\]
运动封闭条件:$\Delta \varphi$等于$2\pi$有理数倍.
\[\Delta \varphi = \int_{r_{min}} ^{r_{max}} \frac{M dr}{\sqrt{2mr^2 [E-U(r)] - \frac{M^2}{m^2 r^2}}} + const.\]
· 有心力场$U(r)$与$\frac{1}{r}\ ,\ r^2$成正比,则运动始终封闭.

\subsection{刚体}
\textbf{刚体}:质点间距离保持不变的质点组成的系统.\\
\textbf{角速度$\vec \omega$}:\\
\textbf{惯性张量}:



\subsection{流体}
\subsubsection{理想流体}
\textbf{连续性方程}
\textbf{Euler方程}
\textbf{Bernoulli方程}
\subsubsection{不可压缩流体}
\subsubsection{粘性流体}
\subsubsection{湍流}
\subsubsection{超流体}



\subsection{振动}



\subsection{正则方法}



\section{相对论力学}
\subsection{相对性原理,相互作用传播速度}
\textbf{相对性原理}: 所有物理定律,在所有惯性参考系中都相同.\\
· 实验表明, 相对性原理是有效的.\\
· 实验表明, 瞬时相互作用在自然界不存在,相互作用的传播需要时间.\\
\textbf{相互作用的传播速度},在所有惯性参考系中都一样(相对性原理可得).\quad 电动力学中证明,这个速度是光在真空中的速度.
\[c = 2.998 \times 10^8 m/s\]
(取$c\to \infty$,即可过渡到经典力学.)\\


\subsection{相对时间}
· <1881年Michelson-Morley干涉实验>表明, 光速与其传播方向无关.\\
(而按经典力学,光应在地球速度同方向(v+c),比反方向(v-c)更快一点.)\\
$\Rightarrow$Galilean变换的绝对时间假设(t=t')错了.\quad  $\Rightarrow$不同参考系,时间流逝的速度不同.\\
\\
\textbf{事件}:由事件发生的位置(x,y,z)和时间(t)决定.\\
\textbf{事件间隔}:
\[S_{12} = [(ct_2-ct_1)^2 - (x_2-x_1)^2 - (y_2-y_1)^2 - (z_2-z_1)^2]^{1/2}\]
$\Rightarrow$ 两个事件的间隔在所有参考系中都一样.\quad 这个不变性,就是光速不变的数学表示.\\
\textbf{固有时}:与物体一同运动的钟所指示的时间.\\
\textbf{固有长度}:物体在相对静止参考系内的长度.


\subsection{参考系间变换}
\textbf{Lorentz变换}: 参考系间变换
\[ x = \frac{x' + V t'}{\sqrt{1 - \frac{V^2}{c^2}}},\quad y=y',\quad z=z'
, \quad t = \frac{t'+ \frac{V}{c^2}x'}{\sqrt{1 - \frac{V^2}{c^2}}}\]
\[ \Rightarrow dx = \frac{dx' + V dt'}{\sqrt{1 - \frac{V^2}{c^2}}},\quad dy=dy',\quad dz=dz'
, \quad dt = \frac{dt'+ \frac{V}{c^2}dx'}{\sqrt{1 - \frac{V^2}{c^2}}}\]
\textbf{速度变换}: $\vec v = \frac{d\vec r}{dt},\quad v' = \frac{d\vec r'}{dt}$
\[\Rightarrow v_x = \frac{v'_x + V}{1 + v'_x \frac{V}{c^2}}, \quad v_y = \frac{v'_y \sqrt{1 - \frac{V^2}{c^2}}}{1 + v'_x \frac{V}{c^2}},\quad v_z = \frac{v'_z \sqrt{1 - \frac{V^2}{c^2}}}{1 + v'_x \frac{V}{c^2}}\]\\
· 例1: \textbf{钟慢}:\\
· 例2: \textbf{尺缩}:\\


\subsection{四维矢量}


\subsection{力学量}
\textbf{Lagrange函数}:
\[L = -m c^2 \sqrt{1 - \frac{v^2}{c^2}}\]
\textbf{动量$\vec P$}:
\[\vec P = \frac{\partial L}{\partial \vec v} = \frac{m \vec v}{\sqrt{1 - \frac{v^2}{c^2}}}\]
\textbf{力$\vec F$}:
\[\vec F = \frac{d\vec P}{dt} \to \frac{m}{\sqrt{1 - \frac{v^2}{c^2}}} \frac{d\vec v}{dt}(\vec F \perp \vec v) 
\quad or\quad  \frac{m}{(1 - \frac{v^2}{c^2})^{3/2}} \frac{d\vec v}{dt} (\vec F \parallel \vec v)\]
\textbf{能量E}:
\[E = \vec P \cdot \vec v - L = \frac{mc^2}{\sqrt{1 - \frac{v^2}{c^2}}}\]
\textbf{静能}:\quad $E(v=0) = mc^2$\\



\section{Electromagnetics 电磁学}
\subsection{电磁场方程}
· 事实表明,粒子同电磁场相互作用的性质,由粒子电荷$q$所决定.\\
\textbf{四维势$A_{i}$: \quad 标势$\varphi$: \quad 矢势$\vec A$:}
\[A^{i}=(\varphi,\vec A)\]
运动方程:
\[\frac{d\vec P}{dt} = - \frac{e}{c} \frac{\partial\vec A}{\partial t} - e \nabla \varphi + \frac{e}{c} \vec v \times \nabla \times \vec A\]
\textbf{电场强度$\vec E$:\quad 磁场强度$\vec H$:}
\begin{displaymath}
    \left\{ \begin{array}{ll}
    \vec E = -\frac{1}{c} \frac{\partial \vec A}{\partial t} - \nabla \varphi\\
    \vec H = \nabla \times \vec A
    \end{array} \right.
\end{displaymath}
\[\Rightarrow \frac{d\vec P}{dt} = e \vec E + \frac{e}{c} \vec v \times \vec H\]
对$\vec E,\vec H$取旋散度, 有\\
\textbf{Maxwell's方程组}:
\begin{displaymath}
    \left\{ \begin{array}{ll}
    \nabla \cdot \vec E = 4\pi\rho\\
    \nabla \cdot \vec H = 0\\
    \nabla \times \vec E = - \frac{1}{c} \frac{\partial \vec H}{\partial t}\\
    \nabla \times \vec H = - \frac{1}{c} \frac{\partial \vec E}{\partial t} + \frac{4\pi}{c}\vec j
    \end{array} \right.
\end{displaymath}


\subsection{特解1: 静电场}


\subsection{特解2: 恒磁场}


\subsection{特解3: 真空电磁波}




\section{Gravitational Field 引力场}



\section{Quantum Mechanics 量子力学}



\section{Statistical Mechanics 统计力学}

\subsection{热力学量}
\textbf{温度 T}:
\textbf{压强 P}:
\textbf{焓}:
\[H = E + P V\]
\textbf{熵}:


\subsection{理想气体}


\subsection{非理想气体}


\subsection{溶液}


\subsection{晶体}


\end{document}
