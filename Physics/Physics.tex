\documentclass{article}
\usepackage{latexsym}
\usepackage[UTF8]{ctex}
\usepackage{geometry}\geometry{left=2cm, right=2cm, top=2cm, bottom=2cm}
\setlength{\parindent}{0pt}
\title{Physics}\author{LiGu 梨菇}\date{}
\begin{document}
\maketitle

\section{Mechanics力学}
\subsection{力学系统描述——运动方程}
(对于s个自由度的系统,)\\
\textbf{广义坐标}: 完全刻画其位置的任意s个变量 $q_{1},q_{2},\cdots,q_{s}$\\
\textbf{广义速度}: $\dot q_{1},\dot q_{2},\cdots,\dot q_{s}$\\
经验表明, 同时给定广义坐标、广义速度, 就可确定系统状态, 原则上可预测以后动作.\\
(加速度$\ddot q$, 可由$\dot q,q$唯一确定.)\\
\textbf{运动方程}: 加速度与坐标、速度的关系式. (二阶微分方程,原则上积分得q(t)可确定系统运动轨迹.)\\
\textbf{Lagrange函数}: 每一个力学系统可以用一个确定函数L()表征.
\[L(q_{1},q_{2},\cdots,q_{s},\dot q_{1},\dot q_{2},\cdots,\dot q_{s},t)\]
\textbf{最小作用量原理}: 系统在两个时刻的, 位置之间的运动, 使得Lagrange函数积分(作用量S)取最小值.
\[S = \int_{t1}^{t2} L(q,\dot q,t)dt\]
\[\delta S = \delta \int_{t1}^{t2} L(q,\dot q,t)dt
 = \int_{t1}^{t2} (\frac{\partial L}{\partial q}\delta q + \frac{\partial L}{\partial \dot q}\delta \dot q) dt
 = \frac{\partial L}{\partial \dot q} \delta q \arrowvert_{t1}^{t2}
 + \int_{t1}^{t2} (\frac{\partial L}{\partial q} + \frac{\partial L}{\partial \dot q}) \delta q dt
 = 0\]
\textbf{Lagrange方程}: 运动微分方程. 
\[\Rightarrow \frac{d}{dt}\frac{\partial L}{\partial \dot q_i} - \frac{\partial L}{\partial q_i} = 0\quad(i=1,\dots,s)\]


\subsection{参考系}
*研究力学现象必须选择参考系.\\
\textbf{惯性参考系}:空间相对它均匀且各向同性,时间相对它均匀.(特别的,某时刻静止的自由物体将永远保持静止.)\\
$\Rightarrow$ Lagrange函数不显含$\vec r,t$,不依赖$\vec v$的矢量方向,即
\[L = L(v^2)\]
$\Rightarrow$ Lagrange方程有$\frac{d}{dt}\frac{\partial L}{\partial \vec v} = -\frac{\partial L}{\partial \vec r} = 0 \Rightarrow \frac{\partial L}{\partial \vec v}=const$\\
\textbf{惯性定律}: 在惯性参考系中,质点任何自由运动的速度的大小和方向都不改变.
\[\Rightarrow \vec v = const\]


\subsection{参考系间相对性}
\textbf{Galilean相对性原理}: 存在无穷个相互作匀速直线运动的惯性参考系,这些惯性系之间时空性质相同,所有力学规律等价.\\
(今后不特别声明,默认惯性参考系.)\\
\textbf{Galilean变换}: 两个不同参考系K、K'之间的的坐标变换(K'相对K以速度$\vec V$运动).\\
\textbf{绝对时间假设}: 我们认为两个参考系的时间相同.
\begin{displaymath}
    \left\{ \begin{array}{ll}
    \vec r = \vec r' + \vec V t\\
    t = t'
    \end{array} \right.
\end{displaymath}
\textbf{Galilean相对性原理}: 力学系统在Galilean变换下具有不变性.


\subsection{守恒定律}
\subsubsection{能量守恒 : 时间均匀性}
时间均匀性: 封闭系统Lagrange函数不显含时间.\quad $\Rightarrow L(\vec q,\dot \vec q)$\\
\[\frac{dL}{dt}
 = \sum \frac{\partial L}{\partial q_i} \dot q_i + \sum \frac{\partial L}{\partial \dot q_i} \ddot q_i
 = \sum \frac{d}{dt}(\frac{L}{\dot q_i}\dot q_i)
\Rightarrow \frac{d}{dt}(\sum \dot q_i \frac{\partial L}{\partial \dot q_i} - L) = 0\]
\textbf{能量E: }
\[\Rightarrow E = \sum \dot q_i \frac{\partial L}{\partial \dot q_i} - L
 = T(q,\dot q) + U(q)   \qquad \frac{d}{dt} E = 0\]


\subsubsection{动量守恒 : 空间均匀性}
空间均匀性: 空间平移不变性.\\
\[\delta L = \sum \frac{\partial L}{\partial \vec r_i}\cdot \delta \vec r_i
 = \vec \epsilon \cdot \sum \frac{\partial L}{\partial \vec r_i} = 0\quad(\forall \vec \epsilon)
\Rightarrow \sum \frac{\partial L}{\partial \vec r_i}
 = \sum \frac{d}{dt} \frac{\partial L}{\partial \vec v_i}
 = \frac{d}{dt} \sum \frac{\partial L}{\partial \vec v_i} = 0\]
\textbf{动量$\vec p$: }
\[\Rightarrow \vec p = \sum \frac{\partial L}{\partial \vec v_i} = \sum m_i \vec v_i  \qquad \frac{d}{dt}\vec p = 0\]


\subsubsection{角动量守恒 : 空间各向同性}
空间各向同性: 空间旋转不变性.
\begin{displaymath}
\begin{array}{ll}
\delta L = \sum (\frac{\partial L}{\partial \vec r_i} \cdot \delta \vec r_i + \frac{\partial L}{\partial \vec v_i} \cdot \delta \vec v_i)
 = \sum [\dot{\vec p_i} \cdot (\delta \vec \varphi \times \vec r_i) + \vec p_i \cdot ( \delta \vec \varphi \times \vec v_i )]
 = \delta \vec \varphi \cdot \sum (\vec r_i \tims \dot{\vec p_i} + \vec v_i \tims \vec p_i) \\
 \quad\  = \delta \vec \varphi \cdot \frac{d}{dt} \sum \vec r_i \times \vec p_i
 = 0\quad(\forall \vec \varphi)
 \Rightarrow \frac{d}{dt}\sum \vec r_i \times \vec p_i = 0
\end{array}
\end{displaymath}
\textbf{角动量$\vec M$: }
\[\Rightarrow \vec M = \sum \vec r_i \times \vec p_i \qquad \frac{d}{dt}\vec M = 0\]


\subsubsection{$E,\vec p,\vec M$参考系间变换}
不同惯性参考系中(K'相对K以速度$\vec V$运动),\\
E 的变换:
\[E = \frac{1}{2} \sum m_iv_i^2 + U = \frac{1}{2} \sum m_i(\vec v_i + \vec V)^2 + U 
 = \frac{m_c V^2}{2} + \vec V \cdot \sum m_i \vec v'_i + \frac{1}{2} \sum m_i v'_i^2 + U
 = E' + \vec V \cdot \vec p' + \frac{m_c V^2}{2}\]
$\vec p$的变换:
\[\vec p = \sum m_i \vec v_i = \sum m_i \vec v'_i + \vec V \sum m_i = \vec p' + \vec V \sum m_i\]
$\vec M$的变换(K'相对K以速度$\vec V$运动且坐标原点相差$\vec R$):
\[\vec M = \sum m_i \vec r_i \times \vec v_i = \sum m_i (\vec r'_i + \vec R) \times (\vec v'_i + \vec V)
 = \sum m_i( \vec r'_i \times \vec v'_i +  \vec R \times \vec v'_i + \vec r'_i \times \vec V + \vec R \times \vec V)
 = \vec M' + \vec R \times \vec p'_c + m_c \vec r'_c \times \vec V + m_c \vec R \times \vec V\]

\subsection{质心}
\textbf{质心系}:\exists 速度$\vec V$使得系统相对K'静止($\vec p' = 0$),且K'原点为系统质量中心,K'即质心系:
\[\vec V = \frac{\vec p}{\sum m_i} = \frac{\sum m_i \vec v_i}{\sum m_i}\]
\textbf{质心}: 
\begin{displaymath}
    \left\{ \begin{array}{ll}
    m_c = \sum m_i\\
    \vec v_c = \frac{\sum m_i \vec v_i}{\sum m_i}\\
    \vec r_c = \frac{\sum m_i \vec r_i}{\sum m_i}
    \end{array} \right.
\end{displaymath}
质心动量: $\vec p_c = m_c \vec v_c$\\
质心能量: \\
内能$E_{int}$: 整体静止的(质心系内)力学系统的能量,包括系统内相对运动动能 + 相互作用势能.\\
\[E = \frac{m_c V^2}{2} + \vec V \cdot \vec p' |_{\vec p'=0} + E_{int} = E_{int} + \frac{m_c v_c^2}{2}\]
质心角动量: 
\[\vec M = \vec M' + \vec R \times \vec p'_c + m_c \vec r'_c \times \vec V + m'_c \vec R \times \vec V|_{\vec p'_c =\vec r'_c = 0, \vec R =\vec r_c, \vec V =\vec v_c}
 = \vec M_{int} + \vec r_c \times \vec p_c \]


\subsection{刚体运动}
\textbf{刚体}:质点间距离保持不变的质点组成的系统.\\
\textbf{角速度$\vec \omega:$}\\
\textbf{惯性张量:}


\subsection{力学相似性}
* Lagrange函数乘任意常数,不会改变运动方程.
\[\vec r_i \to \alpha \vec r_i, t \to \beta t 
 \Rightarrow \vec v_i = \frac{d\vec r_i}{dt} \to \frac{\alpha}{\beta}\vec v_i,T \to \frac{\alpha^2}{\beta^2}T,U \to  \alpha^k U\]
若$\frac{\alpha^2}{\beta^2} = \alpha ^ k $,即$\beta = \alpha^{1-k/2}$,则Lagrange函数乘const.,运动方程不变.前后运动轨迹相似,只是尺寸不同.\\
且各力学量之比,满足:\quad(l:轨迹线度)
\[\frac{t'}{t} = (\frac{l'}{l})^{1-k/2},\frac{v'}{v} = (\frac{l'}{l})^{k/2},\frac{E'}{E} = (\frac{l'}{l})^k,\frac{M'}{M} = (\frac{l'}{l})^{1+k/2}\]\\
· 例1: 均匀力场,势能与坐标成线性,$\Rightarrow k=1,\Rightarrow \frac{t'}{t} = \sqrt{\frac{l'}{l}}$,重力场自由落体,下落时间平方与初始高度成正比.\\
· 例2: Kepler's第三定律,Newton引力、Coulomb力,势能与两点间距离成反比,$\Rightarrow k=-1,\Rightarrow \frac{t'}{t} = (\frac{l'}{l})^{3/2}$,轨道运动周期的平方与轨道尺寸的立方成正比.\\






\section{相对论力学}
\subsection{相对性原理,相互作用传播速度}
\textbf{相对性原理}: 所有物理定律,在所有惯性参考系中都相同.\\
· 实验表明, 相对性原理是有效的.\\
· 实验表明, 瞬时相互作用在自然界不存在.\\
\textbf{相互作用的传播速度},(相对性原理可推断)在所有惯性参考系中都一样.\quad 电动力学中证明,这个速度是光在真空中的速度.
\[c = 2.998 \times 10^8 m/s\]
(取$c\to \infty$,即可过渡到经典力学.)\\


\subsection{相对时间}
· <1881年Michelson-Morley干涉实验>表明, 光速与其传播方向无关.\\
(而按经典力学,光应在地球速度同方向(v+c),比反方向(v-c)更快一点.)\\
$\Rightarrow$Galilean变换的绝对时间假设(t=t')错了.\quad  $\Rightarrow$不同参考系,时间流逝的速度不同.\\
\\
\textbf{事件}:由三维的位置和事件发生的时间决定.


\subsection{参考系间相对性}



\section{Electromagnetics电磁学}


· 事实表明,粒子同电磁场相互作用的性质,由粒子电荷$q$所决定.
\textbf{四维势$A_{i}$: \quad 标势$\varphi$: \quad 矢势$\vec A$:}
\[A^{i}=(\varphi,\vec A)\]
运动方程:
\[\frac{d\vec p}{dt} = - \frac{e}{c} \frac{\partial\vec A}{\partial t} - e \nabla \varphi + \frac{e}{c} \vec v \times \nabla \times \vec A\]
\textbf{电场强度$\vec E$:\quad 磁场强度$\vec H$:}
\begin{displaymath}
    \left\{ \begin{array}{ll}
    \vec E = -\frac{1}{c} \frac{\partial \vec A}{\partial t} - \nabla \varphi\\
    \vec H = \nabla \times \vec A
    \end{array} \right.
\end{displaymath}
\[\Rightarrow \frac{d\vec p}{dt} = e \vec E + \frac{e}{c} \vec v \times \vec H\]
对$\vec E,\vec H$取旋散度, 有\\
\textbf{Maxwell's方程组}:
\begin{displaymath}
    \left\{ \begin{array}{ll}
    \nabla \cdot \vec E = 4\pi\rho\\
    \nabla \cdot \vec H = 0\\
    \nabla \times \vec E = - \frac{1}{c} \frac{\partial \vec H}{\partial t}\\
    \nabla \times \vec H = - \frac{1}{c} \frac{\partial \vec E}{\partial t} + \frac{4\pi}{c}\vec j
    \end{array} \right.
\end{displaymath}

\end{document}
